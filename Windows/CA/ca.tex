\documentclass{article}
\usepackage{graphicx}
\usepackage{hyperref}
\usepackage{listings}
\usepackage{xcolor}
\usepackage{tikzsymbols}
\usepackage{float}

\lstset{
    basicstyle=\ttfamily,
    backgroundcolor=\color{gray!30},
}

\begin{document}

\graphicspath{ {./Images/} }
\tableofcontents

\section{Introduction}
Words

\section{Prerequisites}
\begin {enumerate}
\item You should have joined the CA to the Domain.
\item You should be logged in as a \textbf{domain administrator}
\end {enumerate}

To become a domain administrator, RDP into one of the Domain Controllers.
Then add a user (probably the administrator) to the Cert Publishers group.

\begin{figure}[H]
        \centering
        \includegraphics[width=1\textwidth]{AddingCertPublisherPermission.png}
        \caption{Adding administrator to the Cert Publishers group}
        \label{fig:AddingCertPublisherPermission}
\end{figure}

\begin{figure}[H]
        \centering
        \includegraphics[width=1\textwidth]{BingOnPermissions.png}
        \caption{Adding administrator to the Cert Publishers group}
        \label{fig:BingOnPermissions}
\end{figure}

After adding the permission, you will need to log in and out of the CA for it to take effect.
Note that the domain administrator is possibly logged into differently than the local administrator.

Make sure to login with the administrator account, as seen in \ref{fig:ExampleLoginNETBIOS}

\begin{figure}[H]
        \centering
        \includegraphics[width=1\textwidth]{BingDomainUserAdminUser.png}
        \caption{How to avoid possible naming conflicts}
        \label{fig:BingDomainUserAdminUser}
\end{figure}

\begin{figure}[H]
        \centering
        \includegraphics[width=1\textwidth]{ExampleLoginNETBIOS.png}
        \caption{Example login, explicitly with the Domain Administrator}
        \label{fig:ExampleLoginNETBIOS}
\end{figure}

Your domain admin account should now have the necessary permissions to configure the CA.

\section{Active Directory Certificate Services}

\begin{figure}[H]
        \centering
        \includegraphics[width=1\textwidth]{ServerManager.png}
        \caption{Clicking the ADCS button}
        \label{fig:ADCS}
\end{figure}

Install Active Directory Certificate Services.


\begin{figure}[H]
        \centering
        \includegraphics[width=1\textwidth]{EnterpriseCA.png}
        \caption{The only important button is the Enterprise CA button. Click it!}
        \label{fig:EnterpriseCA}
\end{figure}

Make sure to select Enterprise CA.

\begin{figure}[H]
        \centering
        \includegraphics[width=1\textwidth]{CAName.png}
        \caption{Selecting the CA name}
        \label{fig:CAName}
\end{figure}

TODO: Figure out the importance of the CA name.
I am not sure the significance of the CA name, but I will update this section once I do.

\subsection{Confirming the CA works}

To check the CA we need to make a certificate signing request. 
Unless you have someone who needs a cert, you can make a request by putting this in a text file called "example.inf"

\begin{lstlisting}[breaklines=true, columns=fullflexible]
[Version]
Signature="$Windows NT$"

[NewRequest]
Subject = "CN=example.com"
KeySpec = 1
KeyLength = 2048
Exportable = TRUE
MachineKeySet = TRUE
SMIME = False
PrivateKeyArchive = FALSE
UserProtected = FALSE
UseExistingKeySet = FALSE
ProviderName = "Microsoft RSA SChannel Cryptographic Provider"
ProviderType = 12
RequestType = PKCS10
KeyUsage = 0xa0

[RequestAttributes]
CertificateTemplate = WebServer

[EnhancedKeyUsageExtension]
OID=1.3.6.1.5.5.7.3.1 ; Server Authentication
\end{lstlisting}

\begin{figure}[H]
        \centering
        \includegraphics[width=1\textwidth]{MakingATestCertReq.png}
        \caption{Making a test certificate request}
        \label{fig:MakingATestCertReq}
\end{figure}

Open a PowerShell prompt, navigate to the directory where you saved the example.inf file, 
and run the following command
\begin{lstlisting}[breaklines=true, columns=fullflexible]
certreq -new example.inf example.csr
\end{lstlisting}

\begin{figure}[H]
        \centering
        \includegraphics[width=1\textwidth]{SubmitNewRequest.png}
        \caption{Submitting the request to the CA}
        \label{fig:SubmitNewRequest}
\end{figure}

Open up certsrv, either by searching for "Certiicate Authority" on the machine,
or by doing meta + r to run certsrv.msc.

\end{document}

